\documentclass[11pt,letterpaper]{article}

\input{preamble}

\title{Memo: An \auke-wide initiative on Autonomous Systems}
\author{% \textsf{\large{Kanna Rajan}}\\
  % \emph{Kanna.Rajan@rand.org}\\
  % \url{https://kanna.rajan.systems
  }
\begin{document}

\maketitle{}

\subsection{Background}

\begin{wrapfigure}{!h}{3.8in}
  \vspace{-0.5cm}
  \centering
  \includegraphics[scale=0.06]{fig/word-bag.jpg}
  \caption{Autonomous systems' use is across a range of domains and
    subject matter many of which are cross-cutting within \orge.}
 \label{fig:topics}
\end{wrapfigure}

Autonomous unmanned systems have made steady progress in their
capabilities driven in large part by the ubiquity and scaling effect
of sensors and computational platforms. Starting as niche experimental
platforms confined to structured laboratory environments, they have
not only been in used in space, but increasingly in operational
real-world environments relevant to militaries the world over
demonstrated as never before in the Ukraine conflict. Such systems are
being introduced across multiple domains and subject areas
(Fig. \ref{fig:topics}) across aerial, on land and water, as well as
underwater (Fig. \ref{fig:inverse}). To these we include the
increasingly contested space domain, where low-cost (in the millions
of \$'s) small satellites are playing significant roles in remote
sensing and intelligence gathering.

\begin{wrapfigure}{!h}{3.2in}  
  \centering 
  \includegraphics[scale=0.04]{fig/inverse-pyramid.jpg} 
  \caption{A visualization of autonomous systems and their tools for
    exploration of our natural environment.}
  \label{fig:inverse}
\end{wrapfigure}

Autonomous platforms have been operating in multiple aerial domains
across the world for sometime; they are increasingly operating in the
Persian Gulf, the Black Sea, the Carribean and the South China Sea to
name a few. Land based systems such as the Robotic Combat Vehicle are
planned to be inducted; \noae's use of manned airplanes going thru
Category 4 \& 5 hurricanes for data collection have given way to the
successful use of unmanned surface vehicles.

The primary \emph{modus-operandi} of such systems for now has been via
mixed-initiative (i.e. human-in-the-loop) control. This has allowed
placing the human well outside harms way, while also providing an
extension of the human senses. The technology roadmap of such systems
however, calls for 'dialing up' the autonomy in ways that embedded
machine intelligence (at the core of AI) can make decisions, drive
towards goal achievment and recover from failures, enabling robustness
and consistency in mission operations. Whether they augment the
warfighter, or completely replace them is likely to depend on a range
of issues spanning technology readiness to policy implications of such
self-aware fighting machines. But little thought has been given within
and outside \org to the implication of such robotic warfighting. 

While \org efforts have touched on such system capabilities, the
policy implications related to their resurgance and the implication of
fighting wars and in-turn how such capabilities could impact a range
of issues including foreign policy, defense aquisition,
recruitment/retention all the way to mission capabilities, operations
and posture, have not been adequately studied. We believe it is time
to do so and to get ahead of the 'arc' of such system use and impact.
 
\begin{figure}[!b]
\centering
\vspace{-0.3in}
\subfigure[]{\includegraphics[height=1.68in]{fig/glider.jpg}\label{fig:auv}}
\subfigure[]{\includegraphics[height=1.68in]{fig/DSC_9044.jpeg}\label{fig:lauv}}
\subfigure[]{\includegraphics[height=1.68in]{fig/AutoNaut4.jpg}\label{fig:asv}}
\subfigure[]{\includegraphics[height=1.68in]{fig/uav.jpeg}\label{fig:uav}}
\subfigure[]{\includegraphics[height=1.68in]{fig/titan_in_water.png}\label{fig:rotoruav}}
\caption{ \subref{fig:auv} Glider - an unpropelled Autonomous
  Underwater Vehicle.  \subref{fig:lauv} A propelled Autonomous
  Underwater Vehicle (AUV), deployable from shore or ship.
  \subref{fig:asv} An Autonaut wave-energy driven autonomous surface
  vehicle (ASV).  \subref{fig:uav} A fixed-wing X8 UAV.
  \subref{fig:rotoruav} An Unmanned Aerial Vehicle (UAV) dual-quadcopter.}
\label{fig:systems}
\end{figure}


\subsection{Proposal}

We propose the establishment of an \auke-wide Unmanned System
initiative within \org to position it as a thought leader in the
policy, regulatory and ethical dimensions of the use of unmanned
systems in military and quasi-military (e.g. Coast Guard)
environments. By doing so, \org will tap into its \emph{existing}
expertise in shaping how agencies in the US, UK, the EU and Australia
are working collaboratively from the conceptual use to actual
operations of unmanned systems across space, aerial, surface
(terrestrial and oceanic), underwater and underground domains.

There are three key reasons for such an initiative to span \auke-wide.
One is \orge's own visible footprint in these geographical
entities. Another and equally important reason is the increasing
coordination and collaboration between the entities in \auk
itself. Such an initiative could then be an attractive entity for
sponsors dealing with challenges in scale, culture, operational
environments in addition to the technology itself.  Finally, the
technology induction process across \auk is likely to be similar since
the levels of acceptance, operating methods and maturity across these
regions are similar. Finally, by joining forces across these nations,
\org can rapidly and efficiently help augment the existing
collaborations and connections in the various forces and agencies and
be in a position to rapidly provide 'lessons learned'.

The military space is not the only domain where autonomous systems are
(and likely to) have a lasting impact:

\begin{itemize}

\item The Climate Change requires cafeful, systematic and
  \emph{at-scale} measurement collection and \emph{in-situ}. 70\% of
  the planet is water and many locations hard to reach and/or
  hazardous to operate in. Autonomous systems can go where humans
  often cannot.

\item The oceans have been to-date, a reliable carbon sink; yet our
  understanding of processes and how they're impacted with the
  increasing acidity as a result, has been narrowly focused on corals
  and their bleaching. More exploration (and over vast regions of
  uncharted waters) is needed to understand how the vastness of the
  seas and the life within is changing. This is not only tied to
  food-security of a range of developing coastal nations, but also its
  second-order effects related to shipping security (ref: Somalia,
  Nigeria and the Gulf of Guinea, Yemen).

\item Ditto to the changing atmosphere and the warming it has led to
  over the poles, key areas of climate regulation.

\item Japan and other low-birth rate nations are increasingly relying
  not on immigration, but on robots for elder care and observation. 

\item Climatic impacts to weather resulting in adverse conditions have
  led to a range of adverse events including floods (flash and
  otherwise), tornados, hurricanes, structural failures of man-made
  environments, \ldots.

\item Police and security forces have come to embrace such
  technologies without due regard to implications of their use and the
  data they gather; this will only get worse.


\end{itemize}

As a result, the focus of such an initiative will be on shaping
policies that foster the use and application of autonomous unmanned
systems while ensuring safety, security, and societal acceptance
across a large geographical and political swath.


\subsection{Path forward at \org}

We propose the following implementation plan:

\begin{description}

\item[Establish a core team]: Form a multidisciplinary team of \org
  experts in robotics, AI, sensors, acquisition and operations

\item[Funding and Partnerships]: Secure funding through existing or
  new government grants (i.e. NSRD), private foundations, and industry
  sponsorships subject to review

\item[Project Pipeline]: Identify and initiate high-impact policy
  research projects, prioritizing those with potential to influence
  regulatory frameworks and societal norms
  
\item[Outreach and engagement]: Promote the portfolio’s activities
  through conferences, OpEds, commentaries and media outreach


\end{description}

\subsubsection{Ambition}

Our longer term objective would be a single 'go to' entity within \org
which will span across various divisions, including DPS, SEW, NDRI,
HSRD and the FFRDC's. This entity could be a 'center' which can then
be used as a focal point not just for cross-cutting work in autonomous
systems, but also for chanelling funding including from philanthropies
and non-DoD sponsors, such as \noae, Schmidt Ocean Institute, \ldots.


\end{document}